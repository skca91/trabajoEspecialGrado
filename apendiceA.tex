\chapter{Definici\'on de T\'erminos}

\textbf{Antirretroviral: } Medicamento empleado para impedir la multiplicaci\'on de un retrovirus, como el VIH. Por lo general, el t\'ermino se refiere a los medicamentos antirretrovirales contra el VIH.

\textbf{Carga viral: } Cantidad del VIH en una muestra de sangre. Se notifica como el n\'umero de copias de ARN del VIH por mil\'imetro de sangre. Una meta importante del tratamiento antirretroviral (TAR) es reducir la concentraci\'on de carga viral de una persona a un nivel indetectable, que es demasiado baja para detectar el virus con una prueba de la carga viral.

\textbf{Coinfecci\'on: } es un t\'ermino empleado cuando una persona tiene dos o m\'as enfermedades infecciosas a la vez. Puesto que es posible infectarse con VIH y hepatitis C (VHC o Hep C para abreviar) por las mismas v\'ias,  hasta 3 personas de cada 10 infectadas con el VIH est\'an infectadas tambi\'en con el VHC.

\textbf{Control de versiones: } Es la gesti\'on de los diversos cambios que se realizan sobre los elementos de alg\'un producto o una configuraci\'on del mismo, es decir, a la gesti\'on de los diversos cambios que se realizan sobre los elementos de alg\'un producto o una configuraci\'on.

\textbf{Comorbilidad: } Tambi\'en conocida como "morbilidad asociada", es un t\'ermino utilizado para describir dos o m\'as trastornos o enfermedades que ocurren en la misma persona. Pueden ocurrir al mismo tiempo o uno despu\'es del otro. La comorbilidad tambi\'en implica que hay una interacci\'on entre las dos enfermedades que puede empeorar la evoluci\'on de ambas.

\textbf{DESCRIPTION: } Archivo que contiene informaci\'on b\'asica sobre el paquete.

\textbf{Estudio longitudinal:} Es un tipo de diseño de investigaci\'on que consiste en estudiar y evaluar a las mismas personas por un per\'iodo prolongado de tiempo.

\textbf{Git: } Es un software de control de versiones diseñado por Linus Torvalds.

\textbf{GitHub: } Es una plataforma de desarrollo colaborativo de software para alojar proyectos utilizando el sistema de control de versiones Git.

\textbf{ggplot2:} El paquete ggplot2 de R proporciona un poderoso sistema que hace que sea f\'acil de producir gr\'aficos complejos de varias capas, automatiza varios aspectos tediosos del proceso de graficar manteniendo al mismo tiempo la habilidad de construir paso a paso un gr\'afico pues se compone de una serie de pequeños bloques de construcci\'on independientes, esto reduce la redundancia dentro del c\'odigo, y hace que sea f\'acil de personalizar el gr\'afico para obtener exactamente lo que se desea.

\textbf{IDE: } Una Infraestructura de Datos Espaciales (IDE) es un sistema de informaci\'n integrado por un conjunto de recursos (cat\'alogos, servidores, programas, datos, aplicaciones, p\'aginas Web,…) dedicados a gestionar Informaci\'on Geogr\'afica (mapas, ortofotos, im\'agenes de sat\'elite, top\'onimos,…), disponibles en Internet, que cumplen una serie de condiciones de interoperabilidad (normas, especificaciones, protocolos, interfaces,…), y que permiten que un usuario, utilizando un simple navegador, pueda utilizarlos y combinarlos seg\'un sus necesidades.

\textbf{influence.ME: } Calcula las medidas de influencia para m\'odelos de efectos mixtos estimados con lme4. El razonamiento b\'asico detr\'as de la medici\'on de casos influyentes, es que cuando se omiten iterativamente unidades individuales
a partir de los datos, los m\'odelos basados en estos datos no deber\'ian producir estimaciones sustancialmente diferentes. 

\textbf{ITS: } son enfermedades infecciosas, que pueden transmitirse de una persona a otra durante una relaci\'on sexual vaginal, anal u oral. Las ITS afectan a todos independientemente de la orientaci\'on sexual o identidad de g\'enero. Desde el comienzo de la vida sexual puede estar expuesto/a a estas infecciones. Las producen m\'as de 30 diferentes tipos de virus, bacterias y par\'asitos. Las m\'as frecuentes son la s\'ifilis, gonorrea, clamidia, herpes, hepatitis B y C, VIH y VPH.

\textbf{IAHULA: } Instituto aut\'onomo hospital universitario de los andes.

\textbf{Linfocito $T^{+}CD4$:} Los linfocitos (las c\'elulas) $T^{+}CD4$ ayudan a coordinar la respuesta inmunitaria al estimular a otros inmunocitos, como los macr\'ofagos, los linfocitos B y los linfocitos $T^{+}CD8$ para combatir la infecci\'on. El VIH debilita el sistema inmunitario al destruir los linfocitos CD4.

\textbf{Linfocito $T^{+}CD8$: } Los linfocitos (las c\'elulas) $T^{+}CD8$ reconocen y destruyen las c\'elulas infectadas por microorganismos, como bacterias o virus. 

\textbf{lme4: } Es un paquete de R que proporciona funciones para ajustar y analizar m\'odelos mixtos lineales, m\'odelos mixtos lineales generalizados y m\'odelos mixtos no lineales. En cada uno de estos nombres, el t\'ermino "mixto" o, m\'as completamente, "efectos mixtos", denota un m\'odelo que incorpora t\'erminos de efectos fijos y aleatorios en una expresi\'on predictiva a partir de la cual se puede evaluar la media condicional de la respuesta. 

\textbf{lmerTest: } Proporciona valores de p en anova de tipo I, II o III y tablas de resumen para ajustes del m\'odelo de lmer a trav\'es del m\'etodo de grados de libertad de Satterthwaite. Un m\'etodo de Kenward-Roger tambi\'en est\'a disponible a trav\'es del paquete pbkrtest. Los m\'etodos de selecci\'on de modelos incluyen tablas de pasos, drop1 y anova para efectos aleatorios (ranova). Tambi\'en están disponibles m\'etodos para medios m\'inimos cuadrados (LS-means) y pruebas de contrastes lineales de efectos fijos.

\textbf{M\'odelo lineal mixto: } Es una herramienta flexible para ajustar otros m\'odelos que se pueden formular como m\'odelos lineales mixtos. Dichos m\'odelos incluyen modelos multinivel, modelos lineales jer\'arquicos y modelos de coeficientes aleatorios.

\textbf{NAMESPACE: } Es una de las partes fundamentales de la construcci\'on de un paquete. Tener un NAMESPACE de alta calidad ayuda a encapsular el paquete y hacerlo aut\'onomo. Esto garantiza que otros paquetes no interfieran con el c\'odigo, que el c\'odigo no interfiera con otros paquetes y que el paquete funcione independientemente del entorno en el que se ejecute.

\textbf{nlme: } Ajusta y compara los m\'odelos gaussianos de efectos mixtos lineales y no lineales.

\textbf{Paquete R: } Es una colecci\'on de funciones, datos y c\'odigo R que se almacenan en una carpeta conforme a una estructura bien definida, f\'acilmente accesible para R.

\textbf{plotly: } Es una biblioteca gr\'afica de R, hace gr\'aficos interactivos de calidad de publicaci\'on en l\'inea. Ejemplos de c\'omo hacer gr\'aficos de l\'ineas, diagramas de dispersi\'on, gr\'aficos de \'areas, gr\'aficos de barras, barras de error, diagramas de cajas, histogramas, mapas de calor, subtramas, gr\'aficos de ejes m\'ultiples y gr\'aficos 3D (basados en WebGL).

\textbf{R: } Es un lenguaje orientado a objetos, el c\'alculo estad\'istico y la generaci\'on de gr\'aficos, que ofrece una gran variedad de t\'ecnicas estad\'isticas y gr\'aficas. Es un entorno de an\'alisis y programaci\'on estad\'isticos que, en su aspecto externo, es similar a S. Es un lenguaje de programaci\'on completo con el que se añaden nuevas t\'ecnicas mediante la definici\'on de funciones.

\textbf{Reactividad: } Cuando una entrada (input) cambia, el servidor reconstruye cada salida (output) que depende de ella (tambi\'en si la dependencia es indirecta).

\textbf{REML: } M\'axima verosimilitud restringida.

\textbf{Retrovirus: } Tipo de virus que emplea el ARN como su material gen\'etico. Despu\'es de infectar una c\'elula, un retrovirus emplea una enzima llamada transcriptasa inversa para convertir el ARN en ADN. Luego, el retrovirus integra su ADN en el ADN de la c\'elula huésped, que le permite multiplicarse. El VIH, causante del SIDA, es un retrovirus.

\textbf{RStudio: } Es una interfaz que permite acceder de manera sencilla a toda la potencia de R, para utilizar RStudio se requiere haber instalado R previamente.

\textbf{Seroconversi\'on: } Transici\'on de la infecci\'on por el VIH a la presencia detectable de anticuerpos contra ese virus en la sangre. Cuando ocurre seroconversi\'on (por lo general, a las pocas semanas de la infecci\'on), el resultado de una prueba de detecci\'on de anticuerpos contra el VIH cambia de seronegativo a seropositivo.

\textbf{Shiny: } Es una herramienta para crear f\'acilmente aplicaciones web interactivas (apps) que permiten a los usuarios interactuar con sus datos sin tener que manipular el c\'odigo.

\textbf{SIDA: } Es la enfermedad que se desarrolla como consecuencia de la destrucci\'on progresiva del sistema inmunitario (de las defensas del organismo), producida por un virus descubierto en 1983 y denominado Virus de la Inmunodeficiencia Humana (VIH). La definen alguna de estas afecciones: ciertas infecciones, procesos tumorales, estados de desnutrici\'on severa o una afectaci\'on importante de la inmunidad. La palabra SIDA proviene de las iniciales de S\'indrome de Inmunodeficiencia Adquirida, que consiste en la incapacidad del sistema inmunitario para hacer frente a las infecciones y otros procesos patol\'ogicos. El SIDA no es consecuencia de un trastorno hereditario, sino resultado de la exposici\'on a una infecci\'on por el VIH, que facilita el desarrollo de nuevas infecciones oportunistas, tumores y otros procesos. Este virus permanece latente y destruye un cierto tipo de linfocitos, c\'elulas encargadas de la defensa del sistema inmunitario del organismo.
 
\textbf{VIH: } El virus de la inmunodeficiencia humana (VIH) infecta a las c\'elulas del sistema inmunitario, alterando o anulando su funci\'on. La infecci\'on produce un deterioro progresivo del sistema inmunitario, con la consiguiente "inmunodeficiencia". Se considera que el sistema inmunitario es deficiente cuando deja de poder cumplir su funci\'on de lucha contra las infecciones y enfermedades. El s\'indrome de inmunodeficiencia adquirida (SIDA) es un t\'ermino que se aplica a los estadios m\'as avanzados de la infecci\'on por VIH y se define por la presencia de alguna de las m\'as de 20 infecciones oportunistas o de c\'anceres relacionados con el VIH. El VIH puede transmitirse por las relaciones sexuales vaginales, anales u orales con una persona infectada, la transfusi\'on de sangre contaminada o el uso compartido de agujas, jeringuillas u otros instrumentos punzantes. Asimismo, puede transmitirse de la madre al hijo durante el embarazo, el parto y la lactancia.

\textbf{Wireframe: } Un wireframe o prototipo no es m\'as que un boceto donde se representa visualmente, de una forma muy sencilla y esquem\'atica la estructura de una p\'agina web.
    
