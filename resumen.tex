\chapter*{}
\pagestyle{empty}
\thispagestyle{empty}

\begin{center}
\vspace*{-4cm}
Universidad Nacional Experimental del T\'achira\\
Vicerrectorado Acad\'emico\\
Decanato de Docencia\\
Departamento de Ingenier\'ia Inform\'atica\\

\end{center}
\begin{center}
\textbf{Desarrollo de una aplicaci\'on Web para el estudio de datos longitudinales para el seguimiento de la infecci\'on por el Virus de Inmunodeficiencia Humana, caso de estudio: Instituto Aut\'onomo Hospital Universitario de los Andes}
\end{center}
\begin{flushright}
Autor: Correa Alc\'antara, Stephanie Katherine \\
Tutor:  Timaure Garc\'ia, Rossana. PhD. \\
Fecha: Julio 2018.\\
\end{flushright}

\small{
\begin{center}
\textbf{RESUMEN}
\end{center}

\noindent
Se describe el desarrollo de una aplicaci\'on web como paquete del lenguaje de programaci\'on R,  para el seguimiento de la infecci\'on por el virus de inmunodeficiencia humana (VIH), haciendo uso de datos longitudinales y el modelo lineal mixto. La metodolog\'ia de desarrollo de  \textit{software} empleada fue la espiral, mediante la cual se realiz\'o el filtrado, descripción y exploraci\'on de la base de datos del Laboratorio de Investigaciones Hormonales de Instituto Aut\'onomo Hospital Universitario de los Andes, en la cual se almacena la informaci\'on de seguimiento del comportamiento de los biomarcadores de los pacientes pertenecientes al Programa Nacional SIDA/ITS, de aqu\'i se obtuvieron las vistas minables que sirvieron para determinar el formato de los datos de entrada a la aplicaci\'on;  de igual forma se contruyeron y codificaron los modelos de funciones que integran el paquete, utilizando la dependencia con el paquete  \textit{Shiny} de R para el despliegue gr\'afico sobre el mapa de Venezuela de las caracter\'isticas demogr\'aficas y densidad de los pacientes registrados en la base de datos, as\'i como se implemento una funci\'on para el ajuste y estimaci\'on de un modelo lineal de efectos mixtos, que permita explorar de forma descriptiva, que factores resultan significativos para explicar la variabilidad de la carga viral plasm\'atica \'o el conteo de c\'elulas $T^{+}CD4$. \\

\noindent
\textbf{Palabras clave:} Datos longitudinales, modelo lineal mixto, VIH, R, Shiny.
}


