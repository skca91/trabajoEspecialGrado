
\chapter*{Introducci\'on}
\pagenumbering{arabic} % para empezar la numeración con números

El \textit{software} R es un entorno de c\'odigo abierto para la computaci\'on y graficaci\'on estad\'istica. El \textit{software} se compila y ejecuta en Windows, Mac OS X y Linux, distribuy\'endose usualmente en formato binario para facilitar su instalaci\'on. El proyecto de \textit{software} R fue iniciado por Robert Gentleman y Ross Ihaka. En R, la unidad fundamental de c\'odigo compartible es el paquete o la librer\'ia, el cual agrupa c\'odigo, datos, documentaci\'on y pruebas, y resulta simple de compartir con otros. Para enero del 2015 ya hab\'ian m\'as de 6.000 paquetes disponibles en la Red Integral de Archivos de R, conocido com\'unmente por su acr\'onimo CRAN, el cual es el repositorio de paquetes. Esta gran variedad de paquetes es una de las razones por las cuales R es tan exitoso, pues es probable que alg\'un investigador o acad\'emico ya haya resuelto un problema en su propio campo usando esta herramienta, por lo que otros usuarios simplemente podr\'an recurrir a ella para su uso directo o para llamarla en un nuevo c\'odigo.\\

	El diagn\'ostico de una enfermedad basado en la historia cl\'inica presentada por el paciente es un proceso complejo, involucra s\'intomas espec\'ificos as\'i como estudios realizados y afecciones presentadas en el tiempo. Una de las enfermedades con m\'as incidencia en la humanidad ha sido el Virus de Inmunodeficiencia Humana (VIH), el agente al entrar en el torrente sangu\'ineo por cualquiera de las formas de transmisi\'on, provoca un proceso de infecci\'on, en cierto tipo de c\'elulas del sistema inmunol\'ogico. Estas c\'elulas generan anticuerpos producidos como una reacci\'on de contraataque ante la presencia del VIH,  como los linfocitos $T^{+}CD4$; seg\'un los an\'alisis cl\'inicos dichas c\'elulas deb\'en estar en un rango normal de 500 a 1600 c\'elulas por mil\'imetro cubico de sangre ($mm^{3}$). Cuando estas cifras bajan significativamente a 200 c\'elulas de $T^{+}CD4$ por cada $mm^{3}$ de sangre, se dice que el sistema inmunol\'ogico est\'a muy d\'ebil indicando el desarrollo del Sindrome de Inmunodeficiencia Adquirida (SIDA).\\  

	En ese orden de ideas, la epidemia del VIH presenta muchos factores de gran inter\'es para ser estudiados y analizados bajo un enfoque temporal. En tal sentido, en el estudio epidemiol\'ogico se tiene como objetivo principal descifrar las relaciones que representan las variables persona y tiempo; este \'ultimo muchas veces no es considerado importante por lo tanto implica su estudio, dichas razones dan espacio para el an\'alisis del patr\'on temporal en eventos de salud sea fundamental para entender la exposici\'on al virus y prevenir eventos en el futuro, m\'as all\'a de si el proceso es contagioso, influenciado por el ambiente o relacionado a la variabilidad genot\'ipica. \\
	
	Desde el punto de vista temporal, la epidemiolog\'ia aporta grandes aspectos como lo son la determinaci\'on de los factores y los marcadores biol\'ogicos asociados a la progresi\'on y la efectividad de los tratamientos en la enfermedad, a lo largo del tiempo, por ende es de gran importancia en el enfoque para el estudio de datos longitudinales para el seguimiento de la infecci\'on \\

	La metodolog\'ia de investigaci\'on a aplicar es la de desarrollo de software en espiral, la cual consiste en la evaluaci\'on  de  los riesgos establecidos previamente en la planificaci\'on  y el an\'alisis de los datos por utilizar, en esta ocasi\'on son representativos de cada uno de los valores expuestos y obtenidos a trav\'es de ex\'amenes cl\'inicos realizados a cada uno de los pacientes en un determinado per\'iodo, dicha investigaci\'on  refleja un gran aporte cient\'ifico y m\'edico, con un enorme grado de importancia para la medicina, por lo tanto actualmente no se utiliza ning\'un sistema para el manejo y an\'alisis de datos. \\

	No obstante en cada uno de los cap\'itulos que conforman esta investigaci\'on, se expone y describe de forma detallada el problema del caso de estudio, el objetivo principal y espec\'ificos, la justificaci\'on y su alcance, acompa\~nados con el marco te\'orico, englobando todos aquellos conceptos y teor\'ias indispensables para la mejor comprensi\'on del tema, como tambi\'en el apartado metodol\'ogico empleado y el cual se llev\'o a cabo para la completa resoluci\'on de la investigaci\'on, por \'ultimo el desarrollo y ejecuci\'on del proyecto.
